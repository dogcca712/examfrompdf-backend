\documentclass[12pt]{article}

% ---- Fixed packages (do not change dynamically) ----
% 中文支持:如果内容包含中文,ctex会自动处理;如果不包含,pdflatex也能正常编译
\usepackage[UTF8]{ctex} % 中文支持(pdflatex会忽略,xelatex会使用)
\usepackage[T1]{fontenc}
\usepackage{lmodern} % Latin Modern: paper-like look
\usepackage{amsmath,amssymb}
\usepackage{geometry}
\usepackage{enumitem}
\usepackage{needspace}
\usepackage{fancyhdr}
\usepackage{setspace}
\usepackage{microtype}
\usepackage{tabularx} % 方案A用来排版题干+marks

% ---- Fixed page setup ----
\geometry{a4paper, margin=25mm}
\setstretch{1.15}

% ---- Fix fancyhdr headheight warning ----
\setlength{\headheight}{15pt}

% ---- Header & footer ----
\pagestyle{fancy}
\fancyhf{} % Clear all header and footer fields

% 左上角:Course name;右上角:Page X;中上:Exam title
\lhead{((( course_name )))} 
\chead{((( exam_title )))} 
\rhead{Page \thepage}

\cfoot{} % 页脚留空

% ---- 首行标题 + 课程/日期/名字区域(恢复原样) ----
\newcommand{\ExamTitle}[1]{%
  \begin{center}
    {\Large\bfseries #1\par}
  \end{center}
  \vspace{0.5em}
}

% 这里包含:Course / Date / Name / Student ID
\newcommand{\ExamMeta}[2]{%
  % Course / Date 一行
  \noindent Course: #1 \hfill Date: #2\\[0.75em]
  % Name / Student ID 一行,带下划线
  \noindent Name: \rule{10em}{0.4pt}\hspace{2em}%
  Student ID: \rule{10em}{0.4pt}\\[1.0em]
}

% ---- Question block: ensure enough space to avoid awkward page breaks ----
\newcommand{\QBlockStart}[1]{\needspace{#1\baselineskip}}

% ---------------- Question macros ----------------
% 通用:题号 + 题干 + 分值放在一行里,题干在左列自动换行,marks 固定在右侧
\newcommand{\QHeading}[3]{%
  \QBlockStart{10}%
  \noindent
  \begin{tabularx}{\linewidth}{@{}Xr@{}}%
    \textbf{Q#1.} #2 & \textbf{[#3~marks]} \\
  \end{tabularx}\\[-0.2em]%
}

% MCQ args:
%  #1 q number
%  #2 stem
%  #3 option A
%  #4 option B
%  #5 option C
%  #6 option D
%  #7 marks
\newcommand{\MCQ}[7]{%
  \QHeading{#1}{#2}{#7}%
  \begin{enumerate}[label=(\Alph*), itemsep=0.2em, topsep=0.2em, leftmargin=2.2em]
    \item #3
    \item #4
    \item #5
    \item #6
  \end{enumerate}
  \vspace{0.6em}
}

% SAQ: #1 q number, #2 stem, #3 marks
\newcommand{\SAQ}[3]{%
  \QHeading{#1}{#2}{#3}%
  \vspace{8.5em}
}

% LQ: #1 q number, #2 stem, #3 marks
\newcommand{\LQ}[3]{%
  \QBlockStart{14}%
  \noindent
  \begin{tabularx}{\linewidth}{@{}Xr@{}}%
    \textbf{Q#1.} #2 & \textbf{[#3~marks]} \\
  \end{tabularx}\\[-0.2em]%
  \vspace{12.5em}
}

\begin{document}

% ---- 标题 + 课程/日期/姓名区域(第一页上半部) ----
\ExamTitle{((( exam_title )))}
\ExamMeta{((( course_name )))}{((( exam_date )))}

\noindent\textbf{Instructions}
\begin{itemize}[itemsep=0.2em, topsep=0.2em]
  \item Answer all questions.
  \item This paper is for practice only.
  \item Show your working where appropriate.
\end{itemize}

\vspace{0.8em}

% ---------------- Section A ----------------
\noindent\Large\textbf{Section A: Multiple Choice}\normalsize\\
\noindent\textit{Choose the best answer for each question.}\\
\vspace{0.5em}

((* for q in mcq_questions *))
\MCQ{((( loop.index )))}%
{((( q.stem )))}%
{((( q.options[0] )))}%
{((( q.options[1] )))}%
{((( q.options[2] )))}%
{((( q.options[3] )))}%
{((( q.marks )))}%
((* endfor *))

\vspace{0.6em}

% ---------------- Section B ----------------
\noindent\Large\textbf{Section B: Short Answer}\normalsize\\
\setcounter{enumi}{0}
\vspace{0.5em}

((* for q in saq_questions *))
\SAQ{((( mcq_count + loop.index )))}{((( q.stem )))}{((( q.marks )))}%
((* endfor *))

% ---------------- Section C ----------------
\noindent\Large\textbf{Section C: Long Question}\normalsize\\
\vspace{0.5em}

((* for q in lq_questions *))
\LQ{((( mcq_count + saq_count + loop.index )))}{((( q.stem )))}{((( q.marks )))}%
((* endfor *))

\end{document}
